% Conteúdo do capitulo
%1 - Descrição do desafio experimental para o sphenix
%2 - Desafio de estabelecer o setup experimental
%3 - Desafio de melhorar o sensor LGAD
%4 - Desafio de aplicar para a medida de raios-x
%5 Considerações finais
\chapter{Desafios científicos e tecnológicos}

% 1 - Descrição do desafio experimental para o 
Como descrito anteriormente no texto desta proposta, o desafio científico deste projeto é desenvolver um detector semicondutor para a região de pseudo-rapidez central que seja capaz de melhorar a precisão na medida e na reconstrução de partículas no experimento sPHENIX. 

Para superar esse desafio científico, este projeto irá trabalhar na pesquisa e desenvolvimento do sistema de detecção MVTX, cuja a técnica experimental é baseada na medida da energia depositada nos pixel que compõem o detector, tornando dessa forma possível associar as partículas ao seu vértice de produção para colisões ouro-ouro no RHIC. Para realizar a construção do MVTX, os sensores do tipo MAPS serão adotados como base tecnológica, tendo em vista que eles apresentam excelentes características com respeito ao ganho em altas taxas de colisão. 

% FASE 1
%2 - Desafio de estabelecer o setup experimental
A fase inicial do projeto terá como foco o desenvolvimento da metodologia experimental necessária para a caracterização dos sensores MAPS. Isso será feito através da implantação de técnicas  com o objetivo de caracterizar os sensores MAPS em termos de sua corrente de fuga, ganho, uniformidade e resolução temporal.

Com as técnicas experimentais estabelecidas para a medida da:

\begin{itemize}
\item Corrente de fuga
\item Ganho e uniformidade do ganho
\item Resolução temporal 
\end{itemize}
será possível diagnosticar com precisão a qualidade dos sensores MAPS.

%3 - Desafio de melhorar o sensor LGAD
Uma vez consolidada os experimentos descritos acima para a caracterização dos sensores MAPS, o passo seguinte será continuar com o desenvolvimento do dispositivo, e com base nos dados coletados buscar discutir e optimizar os parâmetros do sensor com o objetivo de produzir novas gerações de MAPS com melhores características em termos de ganho, resolução temporal, uniformidade e resistência à radiação. 

Como dito anteriormente, o programa de pesquisa proposto neste projeto será em grande parte baseado na interação com a colaboração sPHENIX, de modo que teremos acesso aos recentes avanços obtidos no desenvolvimento do MASP, tais como a identificação de vulnerabilidades presentes na geometria do sensor e a degradação do material semicondutor devido ao dano radioativo, os quais serão também tópicos para investigação neste projeto.

Com a diminuição do ganho devido ao dano radioativo, ocorre um aumento da corrente consumida pelo dispositivo MAPS, o que por sua vez provoca o aumento na potência dissipada. Esse fator influencia o projeto de vários outros componentes que farão parte do detector para que seja possível dimensioná-los de forma adequada para comportar a carga de calor produzida pelo dispositivo. Estas questões  demonstram a grande importância de um estudo detalhado desses aspectos durante a fase de pesquisa e desenvolvimento do sensor. 

O design do MVTX encontra-se em seu estágio inicial de desenvolvimento, e desse modo inúmeras oportunidades para contribuição intelectual estão em aberto para serem exploradas na próxima fase da pesquisa e desenvolvimento que está para ser iniciada. 

% CONSIDERACOES FINAIS

Por fim, esta proposta de projeto tem como objetivos estratégicos de curto prazo preparar os experimentos necessários para o desenvolvimento do MAPS, e por conseguinte colaborar com o experimento sPHENIX na construção do MVTX. E para médio e longo prazo os objetivos de desenvolver dispositivos semicondutores para diversas aplicações científicas no âmbito nacional e internacional, e com isso consolidar a tecnologia para a fabricação de sensores semicondutores.

 %varios milestones estabelecidos

%Outro aspecto fundamental   

%Como descrito anteriormente, o ganho em um detector LGAD depende diretamente do perfil da concentração do dopante presente na camada de multiplicação do sensor. Dessa forma, o fator que mais contribui para o dano radioativo em sensores LGAD é a remoção e subsequente redução da concentração de dopante na camada de multiplicação o que por conseguinte provoca a redução do ganho e perdas na resolução em tempo. Com o R&D desenvolvido ate o momento, 

%preliminares demonstram que a adição de outros materiais, tais como carbono, junto com o material dopante podem minimizar o impacto do dano radioativo tornando os sensores mais resistentes à radiação ionizante, entretanto mais estudos devem ser conduzidos para compreender os fatores limitantes obtidos com a adição de outros materiais.


%incluindo a sua produção no Brasil. Vários pesquisadores que integram a equipe deste projeto vêm de outros grupos de pesquisa da USP, do IPEN, conferindo-lhe um forte carácter interdisciplinar, essencial no desenvolvimento de aplicações tanto em imagem de raios X, como na detecção de nêutrons.








