\chapter{Disseminação e avaliação da pesquisa}

Uma vez que parte deste projeto está relacionada com aplicações à detecção de radiação ionizante, essa proposta possui um grande potencial de inovação tecnológica e poderá dar origem a diversos sensores, ampliando os direitos de propriedade intelectual dos processos e dispositivos produzidos no grupo. Além disso, o pesquisador responsável possui experiência no desenvolvimento de instrumentação para física de altas energias, adquirida através do trabalho realizado no projeto de {\it upgrade} do TPC do experimento ALICE \cite{tpcNIM}, isso somado à infraestrutura presente no laboratório do HEPIC para o desenvolvimento de instrumentação nuclear, tornará possível estabelecer e consolidar a linha de pesquisa em sensores semicondutores no Departamento de Física Nuclear do Instituto de Física da USP. Isso irá corroborar com programas experimentais em andamento no grupo relacionado com a espectroscopia e reconstrução de imagens \cite{THGEM,NIM,xray}, ampliando a capacidade tecnológica do HEPIC com respeito à detecção de radiação ionizante. 

À vista disso, no âmbito nacional, com as ferramentas criadas será possível colaborar com os programas de instrumentação em diversos projetos de pesquisa presentes no Brasil tais como o Laboratório Nacional de Luz Síncrotron (LNLS), no que diz respeito ao desenvolvimento de sistemas de detecção. De acordo com os estudos levantados pelos pesquisadores do Sirius e descritas em seu projeto \cite{sirius}, {\it 'existe hoje no Brasil uma oportunidade excepcional para o desenvolvimento de expertise na área de detectores híbridos, visando atender às exigências das linhas de luz do Sirius. Futuramente, essa experiência poderá resultar em desenvolvimentos para as áreas médica, industrial e educacional'}. Isso está alinhado com os objetivos desta proposta.

No final do projeto, o {\it know how} adquirido estará consolidado e pronto para dar continuidade ao trabalho independente deste grupo em novas aplicações bem como na geração de novas tecnologias.

Finalmente, com relação ao gerenciamento do projeto, o seu andamento será monitorado por meio de reuniões semanais e apresentação de resultados ao grupo, onde será reunida as informações necessária para ajustar a estratégia de modo a cumprir os objetivos propostos. Por fim, no decorrer do projeto pretende-se produzir artigos técnicos e científicos documentando os avanços obtidos no desenvolvimento dos sensores, aumentando o impacto internacional do grupo de pesquisa e da tecnologia desenvolvida neste projeto.


