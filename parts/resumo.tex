\chapter*{Resumo}

O experimento sPHENIX, atualmente em fase de desenvolvimento, é um experimento científicos de ultima geração dedicado ao estudo das propriedades do plasma de quarks e gluons (QGP). Esse experimento será construído no Relativistic Heavy Ion Collider (RHIC) no Brookhaven National Laboratory (BNL). 

Para atender aos requisitos experimentais, o sPHENIX  MASP-based Vertex Detector (MVTX) permitirá reconstruir os eventos à uma alta taxa de colisão utilizando sensores finos com baixa quantidade de material. Esse sistema de detecção irá fornecer uma alta eficiência na reconstrução de trajetórias e uma excelente resolução na medida do parâmetro de impacto, características ideais para identificar hádrons oriundos de quarks pesados, e reconstrução de eventos para as altas taxas de colisão que serão fornecidas pelo RHIC.

A vista disso, neste documento é proposto o desenvolvimento de atividades de pesquisa em colaboração com o experimento sPHENIX. O trabalho será focado na pesquisa e desenvolvimento dos sensores semicondutores do tipo MAPS - recentemente desenvolvidos para tolerar altos níveis de radiação ionizante - com o objetivo de optimizar sua capacidade para utilização no detector MVTX do experimento sPHENIX.

