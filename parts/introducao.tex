% Conteudo do capitulo
%1- Razoes fisicas para se ter o SPHENIX 
%5- Proposta do projeto
%6- Alguns detalhes sobre o escopo do projeto
%7- Importancia para o grupo 
\chapter{Introdução}

O sPHENIX é um experimento de íons pesados relativísticos que será construído no Relativistic Heavy Ion Collider (RHIC) em New York (USA) para estudar a dependência da perda de energia de quarks pesados, charm e bottom em função de suas propriedades de transporte no plasma de quarks e glúons (QGP), permitindo investigar em detalhes a interação cromodinâmica desses quarks com o meio formado na colisão. Para tanto, será necessário medir com precisão e eficiência as trajetórias das partículas próximas ao ponto de interação, vértice primário, de modo a identificar precisamente os vértices secundários originados do decaimento de mésons oriundos da hadronização de quarks pesados.

Para fazer essa medida com precisão, será desenvolvido um detector de silício denominado MAPS-based Vertex Detector (MVTX), que será composto por Monolithic Active Pixel Sensor (MAPS), os quais são capazes de operar em altas taxas de colisão sem perder a precisão nas medidas das partículas, o que é uma característica requerida para esse tipo de aplicação.    

Com essas características, o detector de vértice MVTX irá permitir que o sPHENIX forneça informações importantes para o estudo da interação de quarks pesados no QGP, através da medida de observáveis tais como a correlação de mésons oriundos de quarks pesados, fator de modificação nuclear R$_{AA}$, fluxo elíptico V$_{2}$, e de ordem superior V$_{n}$, que são observáveis relacionados com as propriedades hidrodinâmicas do QGP.

Desse modo, o sPHENIX será capaz de expandir o conhecimento sobre o QGP, permitindo explorar com maior precisão uma grande variedade de observáveis em função da massa dos quarks pesados \textit{charm} e \textit{bottom} ($c$,$b$), os quais atualmente se constituem nas pontas de provas duras (ou \textit{hard probes}) empregadas no estudo das propriedades do plasma.

O par de quarks \textit{c} e \textit{b} são por sua vez importantes pois são criados predominantemente no espalhamento duro de quarks na fase inicial da colisão, cuja produção é determinada pela teoria da cromodinâmica quântica perturbativa, e participam de toda a evolução do QGP, o que permite extrair informações da interação desses quarks com o meio formado.

%2 proposta do projeto
A vista disso, o propósito deste projeto é desenvolver técnicas e metodologias necessárias para a caracterização física de sensores semicondutores do tipo MAPS, e dos componentes do detector MVTX. Neste escopo a colaboração com o experimento sPHENIX será fundamental pois irá servir de ponte para o intercambio de conhecimento e experiência que serão importantes, permitindo introduzir futuramente esse tipo de tecnologia no Brasil.

O projeto será composto de várias etapas onde inicialmente pretende-se consolidar as técnicas necessárias para a caracterização dos sensores e montagem do detector. Neste processo o pesquisador e colaboradores irão estabelecer os 
métodos experimentais para a caracterização das propriedades físicas dos sensores. Nesta estágio, o pesquisador será responsável por montar e certificar os arranjos experimentais, desenhar e construir as partes necessárias para os mesmos, além de automatizar o processo de tomada de dados. Por fim, a discussão dos resultados obtidos com os centros de pesquisa dentro da colaboração sPHENIX será feita para garantir a qualidade dos resultados obtidos e acelerar sua consolidação.

Por fim, é importante destacar que a implantação dessa linha de pesquisa no grupo HEPIC do Instituto de Física da Universidade de São Paulo será importante para projetos futuros envolvendo sensores semicondutores. Atualmente o grupo conta com ferramentas na área de instrumentação para física nuclear e de partículas reconhecida internacionalmente, e adquirida por meio da execução de projetos em colaborações internacionais, como por exemplo o chip SAMPA focada na tecnologia de detectores gasosos \cite{ref1}. Isso somado à experiência do pesquisador responsável adquirida na execução de pesquisa e desenvolvimento no âmbito do projeto de upgrade do TPC do ALICE \cite{tpcNIM,discharge_paper,GSI_REPO}, tornará possível a transferência total de tecnologia relacionada com sensores semicondutores de alta performance para o HEPIC, bem como a geração de novas tecnologias e aplicações.

\renewcommand{\cleardoublepage}{}
\renewcommand{\clearpage}{}