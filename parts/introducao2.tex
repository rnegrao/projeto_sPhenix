\chapter{Introdução}

O experimento sPHENIX, atualmente em fase de desenvolvimento, se constitui em um experimento científico de ultima geração, dedicado ao estudo das propriedades do plasma de quarks e gluons (QGP). Esse experimento será construído no Relativistic Heavy Ion Collider (RHIC) no Brookhaven National Laboratory (BNL), e está programado para entrar em operação no ano de 2023.

O sPHENIX foi desenhado empregando as mais recentes e eficientes técnicas experimentais existentes para a detecção de partículas, com o objetivo de estudar a natureza da interação forte de quarks pesados com o plasma de quarks e gluon (QGP). 

Neste sentido, grandezas como o fator de modificação nuclear e o fluxo elíptico V$_{2}$ de quarks pesados são observáveis importantes que somados à reconstrução de jatos oriundos de quarks pesados, charm e bottom, oferecem informações fundamentais para o entendimento das propriedades dos QGP, tornando possível estudar a perda de energia desses partons no meio em função da massa dos quarks e temperatura do plasma.

Do ponto de vista experimental, a medida desses observáveis requer alta precisão e eficiência na reconstrução das trajetórias próximas ao ponto de interação, de modo a fornecer uma identificação precisa dos vértices secundários oriundos de quarks pesados. Assim, um detector de vértice com alta velocidade e grande aceitação torna-se necessário para medir esses observáveis.  

Para contornar esse desafio experimental, o sPHENIX irá construir um detector de vértice composto por detectores de silício pixelados, cujos sensores empregarão a tecnologia denominada Monolithic Active Pixel Sensor (MAPS), utilizada em diversos experimentos com altas taxas de colisão. O detector proposto, denominado MASP-based Vertex Detector (MTVX), utilizará sensores MAPS de ultima geração garantindo que o sistema tenha alta performance e seja capaz de efetuar as medidas em rapidez central e altas taxas de colisão, para um grande intervalo de momento transversal.

Possuindo excelentes características , o MVTX irá possibilitar o estudo da física de quarks pesados \textit{c} e \textit{b} no QGP, para as energias disponíveis no RHIC. Esses estudos estão além da capacidade de medida dos experimentos atuais presentes no RHIC, sendo desse modo importantes para o entendimento das propriedades de transporte no QGP nas energias do RHIC.

O detector MVTX fornecerá uma alta taxa de contagem utilizando sensores finos com baixa quantidade de material e espessura de $30\mu m$. Esse sistema de detecção irá fornecer uma alta eficiência na reconstrução de trajetórias e uma excelente resolução na medida do parâmetro de impacto, características ideais para identificar hádrons oriundos de quarks pesados, e reconstrução de eventos para as altas taxas de colisão que serão fornecidas pelo RHIC.

%2 proposta do projeto
A vista disso, o propósito deste projeto é desenvolver técnicas e metodologias necessárias para a caracterização física de sensores semicondutores do tipo MAPS, e dos componentes do detector MVTX. Neste escopo a colaboração com o experimento sPHENIX será fundamental pois irá servir de ponte para o intercambio de conhecimento e experiência que serão importantes, permitindo introduzir futuramente esse tipo de tecnologia no Brasil.

O projeto será composto de várias etapas onde inicialmente pretende-se consolidar as técnicas necessárias para a caracterização dos sensores e montagem do detector. Neste processo o pesquisador e colaboradores irão estabelecer os 
métodos experimentais para a caracterização das propriedades físicas dos sensores. Nesta estágio, o pesquisador será responsável por montar e certificar os arranjos experimentais, desenhar e construir as partes necessárias para os mesmos, além de automatizar o processo de tomada de dados. Por fim, a discussão dos resultados obtidos com os centros de pesquisa dentro da colaboração sPHENIX será feita para garantir a qualidade dos resultados obtidos e acelerar sua consolidação.

Neste ponto é importante destacar que a implantação dessa linha de pesquisa no grupo HEPIC do Instituto de Física da Universidade de São Paulo será importante para projetos futuros envolvendo sensores semicondutores. Atualmente o grupo conta com ferramentas na área de instrumentação para física nuclear e de partículas reconhecida internacionalmente, e adquirida por meio da execução de projetos em colaborações internacionais, como por exemplo o chip SAMPA focada na tecnologia de detectores gasosos \cite{ref1}. Isso somado à experiência do pesquisador responsável adquirida na execução de pesquisa e desenvolvimento no âmbito do projeto de upgrade do TPC do ALICE \cite{tpcNIM,discharge_paper,GSI_REPO}, tornará possível a transferência total de tecnologia relacionada com sensores semicondutores de alta performance para o HEPIC, bem como a geração de novas tecnologias e aplicações.

\renewcommand{\cleardoublepage}{}
\renewcommand{\clearpage}{}